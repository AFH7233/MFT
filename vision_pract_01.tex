\documentclass[conference]{IEEEtran}
\IEEEoverridecommandlockouts%The preceding line is only needed to identify funding in the first footnote. If that is unneeded, please comment it out.

\usepackage[utf8]{inputenc}
\usepackage{cite}
\usepackage{amsmath,amssymb,amsfonts}
\usepackage{algorithmic}
\usepackage{graphicx}
\usepackage{textcomp}
\usepackage{xcolor}
\usepackage{lipsum}


\def\BibTeX{{\rm B\kern-.05em{\sc i\kern-.025em b}\kern-.08em
    T\kern-.1667em\lower.7ex\hbox{E}\kern-.125emX}}

%Para justificar sin cortar palabras
\tolerance=1
\emergencystretch=\maxdimen
\hyphenpenalty=10000
\hbadness=10000

%-------------------------------------------------------------------------------
%	BEGIN DOCUMENT
%-------------------------------------------------------------------------------
\begin{document}

%-------------------------------------------------------------------------------
%	DOCUMENT INFORMATION
%-------------------------------------------------------------------------------
\title{Práctica 1 - MTF del ojo humano}


%-------------------------------------------------------------------------------
%	AUTHORS
%-------------------------------------------------------------------------------
\author{
\IEEEauthorblockN{Daniel Torres Robledo}
\IEEEauthorblockA{\textit{shadow.cat6333@gmail.com}}
\and
\IEEEauthorblockN{Andrés Fuentes Hernández}
\IEEEauthorblockA{\textit{andres7233@hotmail.com}}

}

\maketitle


%-------------------------------------------------------------------------------
%	ABSTRACT
%-------------------------------------------------------------------------------
\begin{abstract}
En este documento se describen las aplicaciones de la \textit{MTF}, el cálculo de ecuaciones para generarla en \textit{python}, así como experimentos utilizando este estímulo con diferentes personas a distintas distancias, que para obtener medidas invariantes a la distancia de observación, el cálculo de esta frecuencia será expresado en ciclos/grado.
\end{abstract}

\begin{IEEEkeywords}
MTF, SVH, python, sinusoidal modulada, experimentos, invariantes
\end{IEEEkeywords}


%-------------------------------------------------------------------------------
%	SECTION 1
%-------------------------------------------------------------------------------
\section{\textit{Objetivo}}


\begin{itemize}
\item Encontrar la \textit{MTF} del ojo experimentalmente.
\item Encontrar la frecuencia de máxima sensibilidad del ojo humano.
\item Generar los cálculos necesarios para obtener la función sinusoidal en frecuencia y amplitud.
\item Desplegar el estímulo visual sinusoidal modulada.
\end{itemize}


%-------------------------------------------------------------------------------
%	SECTION 1
%-------------------------------------------------------------------------------
\section{\textit{Introducción}}

La \textit{MTF} (\textit{Modulation Transfer Function}) o función de transferencia de modulación se ha convertido en una herramienta muy utilizada para especificar el rendimiento y la resolución de toda clase de sistemas ópticos y de visión, que van desde lentes, cintas magnéticas y películas hasta telescopios, la atmósfera y el ojo humano.

La \textit{MTF} es usada para caracterizar sistemas lineales e invariantes en el espacio, sin embargo, a pesar de que el Sistema Visual Humano (SVH) no cumple con estas dos propiedades, la \textit{MTF} se ha utilizado para caracterizarlo bajo condiciones de iluminación controladas.

La \textit{MTF} también puede interpretarse como la capacidad de un sistema óptico para percibir contraste, es decir, la capacidad de resolver o diferenciar líneas a una determinada frecuencia espacial.

Para entender de manera simplificada el concepto de frecuencia espacial, hay que notar que las frecuencias espaciales bajas equivalen a repeticiones de patrones muy separados espacialmente (por ejemplo, líneas muy separadas) y que las altas equivalen a repeticiones más compactas.

Hay que considerar que todo sistema de visión tiene un límite superior a partir del cual ya no le es posible distinguir más detalles. Este límite está directamente relacionado con la frecuencia de \textit{Nyquist}\cite{p4}. En los estudios de sensibilidad del \textit{SVH} a base de experimentación psico-física, se ha encontrado que su \textit{MTF} consiste en una función del tipo paso-banda con un pico o frecuencia espacial máxima en el rango de 2 a 6 ciclos por grado\cite{p1}\cite{p2}.

En esta practica se examinará la sensibilidad del \textit{SVH} a distintas frecuencias espaciales; la idea general de crear un patrón de sinusoidales, cuya frecuencia aumenta en el eje horizontal y el contraste en el eje vertical. La envolvente del patrón visibles generalmente muestra un comportamiento similar a la curva de la \textit{MTF}.

Comprobaremos que dicho pico, que representa la frecuencia máxima de sensibilidad del ojo humano se moverá conforme el observador se aleja o acerca del estímulo. Para obtener una medición invariante a la distancia de observación, el cálculo de esta frecuencia de observación debe ser expresado en ciclos/grado.


%-------------------------------------------------------------------------------
%	SECTION 1
%-------------------------------------------------------------------------------
\section{\textit{Desarrollo}}

En esta sección se muestra el cálculo para obtener las constantes de la función sinusoidal modulada en frecuencia y en amplitud.





\begin{figure}[htbp]
\centerline{\includegraphics[width=80mm]{code/mtf}}
\caption{Estímulo visual (sinusoidal modulada).}
\label{mtf}
\end{figure}


%-------------------------------------------------------------------------------
%	SECTION 1
%-------------------------------------------------------------------------------
\section{\textit{Resultados}}
\textcolor{violet}{Los resultados deberán presentarse con los cálculos respectivos para obtener la frecuencia de máxima sensibilidad para cada una de lasdistancias con las que se experimentó}

3.Observar la MTF a varias distancias de observación.
4.Calcular la frecuencia de máxima sensibilidad (en ciclos/grado) a 3 distancias diferentes de observación


\begin{figure}[htbp]
\centering

\begin{tabular}{c|c|c|c|}
	 & $D_1=30cm$ & $D_2=60cm$ & $D_3=100 cm$\\
	\hline
	$X$ & 0 & 0 & 0\\
	\hline
	$\phi'$ & 0 & 0 & 0\\
	\hline
	$\alpha$ & 0 & 0 & 0\\
	\hline
	$f$ & 0 & 0 & 0\\
	\hline
\end{tabular}

\caption{Tabla de resultados, experimento 1.1.}
\label{res1.1}
\end{figure}


\begin{figure}[htbp]
\centering

\begin{tabular}{c|c|c|c|}
	 & $D_1=30cm$ & $D_2=60cm$ & $D_3=100 cm$\\
	\hline
	$X$ & 0 & 0 & 0\\
	\hline
	$\phi'$ & 0 & 0 & 0\\
	\hline
	$\alpha$ & 0 & 0 & 0\\
	\hline
	$f$ & 0 & 0 & 0\\
	\hline
\end{tabular}

\caption{Tabla de resultados, experimento 1.2.}
\label{res1.2}
\end{figure}



\begin{figure}[htbp]
\centering

\begin{tabular}{c|c|c|c|}
	 & $D_1=30cm$ & $D_2=60cm$ & $D_3=100 cm$\\
	\hline
	$X$ & 0 & 0 & 0\\
	\hline
	$\phi'$ & 0 & 0 & 0\\
	\hline
	$\alpha$ & 0 & 0 & 0\\
	\hline
	$f$ & 0 & 0 & 0\\
	\hline
\end{tabular}

\caption{Tabla de resultados, experimento 1.3.}
\label{res1.3}
\end{figure}




%-------------------------------------------------------------------------------
%	SECTION 1
%-------------------------------------------------------------------------------
\section{\textit{Conclusiones}}
\textcolor{violet}{\lipsum[3]}


%-------------------------------------------------------------------------------
%	SECTION 1
%-------------------------------------------------------------------------------
\section{\textit{Código fuente}}
\textcolor{violet}{\lipsum[3]}



%-------------------------------------------------------------------------------
%	BIBLIOGRAPHY
%-------------------------------------------------------------------------------
\begin{thebibliography}{00}
\bibitem{p1} Pratt, W. k., Digital Image Processing, John Wiley \& Sons Inc, 2001.

\bibitem{p2} Levine, M.D., Vision in man and machine, McGraw-Hill, 1985.

\bibitem{p3} Norman Koren, \textit{Resolution and MTF curves in scanners and sharpening}. Avalible at \textit{http://www.normankoren.com/Tutorials/MTF2.html}. [Accessed Agosto 31, 2019]

\bibitem{p4} Norman Koren, \textit{Nyquist frequency}. Avalible at \textit{https://en.wikipedia.org/wiki/Nyquist\_frequency}. [Accessed Agosto 31, 2019]
\end{thebibliography}







%-------------------------------------------------------------------------------
%	END DOCUMENT
%-------------------------------------------------------------------------------
\end{document}